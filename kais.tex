% sample article for KAIS
% last modified by alison woollatt, 18 jan 00
% minor changes by xindong wu, 2 march 01

\documentclass{kais}
%
\usepackage{wrapfig,epsf,graphicx,mathtools,multirow}
\usepackage[dcucite]{harvard}

\newtheorem{defi}{Definition}[section]
\newtheorem{property}{Property}[section]
\newtheorem{algorithm}{Algorithm}[section]

\newcommand\changenum{%
  \renewcommand\labelenumi{\theenumi}%
  \renewcommand{\theenumi}{(\arabic{enumi})}%
}

\newcommand\changepnum{%
  \renewcommand\labelenumi{\theenumi}%
  \renewcommand{\theenumi}{($P_\arabic{enumi}$)}%
}

%\bibliographystyle{agsm}

\received{xxx}
\revised{xxx}
\accepted{xxx}

\pubyear{2000}
\pagerange{\pageref{firstpage}--\pageref{lastpage}}
\volume{xxx}

\begin{document}
\label{firstpage}

\title{Variability-based improvement of m-learning applications development}

\author[FalvoJr et al]{V.~FalvoJr$^1$, A.~Marcolino$^1$, N.~Duarte~Filho$^1$,E.~OliveiraJr$^2$ and
E.~Barbosa$^1$\\ $^1$University of S\~ao Paulo (USP), S\~ao Paulo, Brazil;\\$^2$State University of Maring\'a (UEM), Paran\'a, Brazil}
 
\maketitle

\begin{abstract}
Software Product Lines (SPL) aim at improving the product quality and time-to-market of the development methodology of singular software. The central concept of such a reuse approach is variability, which provides diversity of software products in a specific domain. In the educational domain, the popularity of mobile devices has contributed to the increasing demand for mobile learning (m-learning) applications. This paper discusses how variability can improve the development of m-learning applications by adoption of SMarty, a concise variability management approach. An SPL for m-learning applications has been proposed and experimentally evaluated in industry, providing evidences that well-defined variabilities can speed up time-to-market of m-learning products with a reduced number of faults.
\end{abstract}

\begin{keywords}
mobile learning applications; software product lines; variability management; experimental evaluation
\end{keywords}

\input{sections/section1}

\input{sections/section2}

\input{sections/section3}

% 30/12/2015 06:22 %

\section{M-SPLear\allowbreak ning Experimental Evaluation}\label{section4}

This section reports the experimental evaluation of M-SPLear\allowbreak ning regarding time-to-market and quality improvement. M-SPLear\allowbreak ning was compared with a software singular development methodology. The guidelines proposed by Jedlitschka et al. \cite{jedlitschka07} were followed for controlled experiments.

\subsection{Motivation}\label{sub:motivation}


The choice for the adoption of new technologies or approaches used in the development process depends on several aspects of quality and benefits. Although many approaches have been developed, many issues about them must be solved for a suitable adoption in both industry and academic environments. Experimental evaluations may shed light on the identification of evidences from quality and benefits of an approach, justifying their choice. In some cases, the collected evidence may still support the correction of problems identified during the experimental evaluations and improvements in the proposals \cite{wohlin12,juristo10}.

Therefore, the experimental evaluation of the M-SPLear\allowbreak ning is presented based on two relevant software development variables: time-to-market and number of faults. These variables can be directly influenced by the adopted development methodology and approaches that support the variabilities and commonalities management \cite{hubaux10,capilla13}. Additionally, in industry, quality and time-to-market, such as other variables, can defined the success or unsuccess of the business to attend satisfactorily their clients \cite{hubaux10}. Therefore, based on the related works (see Section \ref{section6}) and believing that our results complement the already conducted researches, we select these two variables to be experimentally compared in our study, considering the influence in the adoption of a concise variability management approach in the design phase of the M-SPLear\allowbreak ning.


\subsubsection{Research Objectives}\label{sub:object}

The experiment aimed at a \textbf{comparison} between the singular software development (SSD) and the software product line (SPL) methodologies, \textbf{for the purpose of} identification of the most efficient, \textbf{regarding} the time spent on the creation of software products (time) and number of their faults, \textbf{in the context of} practitioners from industry.


\subsection{Experimental Design}\label{sub:design}

This section describes the experimental design and procedures for support of future replications.

\subsubsection{Goals}

Two research questions (R.Q.) based on the research objective were raised:

\textbf{R.Q.1} Which methodology is more efficient regarding time-to-market, SSD or SPL?

\textbf{R.Q.2} Which methodology showed more quality in terms of number of faults in the software product created, SSD or SPL?

\subsubsection{Hypotheses}

Two sets of hypotheses were defined to be tested and each of them is related to its respective research questions (R.Q.1 and R.Q.2):

\textbf{R.Q.1 hypotheses:} time-to-market

	\begin{itemize}
	
	\item \textbf{Null Hypothesis ($H_{0}$)}: there is no significant difference of time-to-market between SSD and SPL.
	
	$H_{0}$ : $\mu$(\textit{t(SSD)}) =  $\mu$(\textit{t(SPL)});
	
	\item \textbf{Alternative Hypothesis ($H_{1}$)}: SSD has less time-to-market than SPL.
	
	$H_{1}$ : $\mu$(\textit{t(SSD)}) $<$ $\mu$(\textit{t(SPL)});
		
	
	\item \textbf{Alternative Hypothesis ($H_{2}$)}: SSD has more time-to-market than SPL.
	
	$H_{2}$ :  $\mu$(\textit{t(SSD)}) $>$ $\mu$(\textit{t(SPL)}).		
	
	\end{itemize}	

\textbf{R.Q.2 hypotheses:}
quality in terms of number of faults
	\begin{itemize}
	
	\item \textbf{Null Hypothesis ($H_{0}$)}: there is no significant difference between SSD and SPL with regard to quality regarding number of faults in the software products created.
	
	$H_{0}$ : $\mu$(\textit{d(SSD)}) =  $\mu$(\textit{d(SPL)});
	
	\item \textbf{Alternative Hypothesis ($H_{1}$)}: SSD has a larger number of faults than SPL.
	
	$H_{1}$ : $\mu$(\textit{d(SSD)}) $>$ $\mu$(\textit{d(SPL)});
		
	\item \textbf{Alternative Hypothesis ($H_{2}$)}: SSD has a smaller number of faults than SPL.
	
	$H_{2}$ :  $\mu$(\textit{d(SSD)}) $<$ $\mu$(\textit{d(SPL)}).		
	
	\end{itemize}

\subsubsection{Variables}

Dependent variables are the mean of time ($t$) and faults ($f$), defined as follows:

\small

\begin{equation}\label{eq:1}
\mu{(t)}=(\Sigma xi)/n, i = 1..n
\end{equation}
\begin{equation}\label{eq:2}
\mu{(f)}=(\Sigma yi)/n, i = 1..n
\end{equation}
\normalsize 
where:
\begin{itemize}
\item \textit{t} is the implementation time (minutes);
\item \textit{f} is the number of faults;
\item \textit{xi} is the time of implementation of participant i;
\item \textit{yi} is the number of faults detected in the implementation of participant i; and
\item \textit{n} is the total of participants in the experiment.
\end{itemize}
\normalsize

Independent variables are the development methodology, which is a factor with two treatments (SSD and SPL), and the software product configuration for mobile learning platform, which is a factor with two treatments namely product 1 (P1) and product 2 (P2). Table \ref{tab:variables} shows the description of dependent and independent variables.

%\begin{landscape}

\begin{table}
\centering
\caption{\label{tab:variables}Dependent and Independent Variables Description.}
\includegraphics[scale=0.75]{figures/section4/tab_n.pdf}



%\resizebox{1.2\textwidth}{!}{
%\small
%\begin{tabular}{cccccccccc}
%\multicolumn{1}{l}{}                                                                         & \multicolumn{1}{l}{}                                                                                                                      & \multicolumn{1}{l}{}                                 & \multicolumn{1}{l}{}                                                                                                        & \multicolumn{1}{l}{}                                                                                           & \multicolumn{1}{l}{}                                                                                                        & \multicolumn{1}{l}{}                                                                                            & \multicolumn{1}{l}{}                                                                & \multicolumn{1}{l}{}                                                                                                                             & \multicolumn{1}{l}{}                                              \\ \hline
%\multicolumn{1}{c|}{\textbf{\begin{tabular}[c]{@{}c@{}}Name of the\\ Variable\end{tabular}}} & \multicolumn{1}{c|}{\textbf{\begin{tabular}[c]{@{}c@{}}Type of the\\ Variable\\ (independent, \\ dependent, \\ moderating)\end{tabular}}} & \multicolumn{1}{c|}{\textbf{Abbreviation}}           & \multicolumn{1}{c|}{\textbf{\begin{tabular}[c]{@{}c@{}}Class\\ (product, \\ process, \\ resource, \\ method)\end{tabular}}} & \multicolumn{1}{c|}{\textbf{\begin{tabular}[c]{@{}c@{}}Entity\\ (instance of \\ the class)\end{tabular}}}      & \multicolumn{1}{c|}{\textbf{\begin{tabular}[c]{@{}c@{}}Type of \\ Attribute \\ (internal, \\ external, \ \ldots)\end{tabular}}} & \multicolumn{1}{c|}{\textbf{\begin{tabular}[c]{@{}c@{}}Scale\\  Type \\ (nominal, \\ ordinal,\ \ldots)\end{tabular}}} & \multicolumn{1}{c|}{\textbf{Unit}}                                                  & \multicolumn{1}{c|}{\textbf{Range}}                                                                                                              & \textbf{\begin{tabular}[c]{@{}c@{}}Counting \\ Rule\end{tabular}} \\ \hline
%\multicolumn{1}{c|}{\begin{tabular}[c]{@{}c@{}}Development\\ methodology\end{tabular}}       & \multicolumn{1}{c|}{\multirow{2}{*}{independent}}                                                                                         & \multicolumn{1}{c|}{DM}                              & \multicolumn{1}{c|}{method}                                                                                                 & \multicolumn{1}{c|}{\begin{tabular}[c]{@{}c@{}}Software\\ development \\ methodology\end{tabular}}             & \multicolumn{1}{c|}{N.A.}                                                                                                   & \multicolumn{1}{c|}{nominal}                                                                                    & \multicolumn{1}{c|}{N.A.}                                                           & \multicolumn{1}{c|}{\begin{tabular}[c]{@{}c@{}}SSD – Singular \\ software \\ development \\ and SPL \\ Software \\ product\\  line.\end{tabular}} & N.A.                                                              \\ \cline{1-1} \cline{3-10} 
%\multicolumn{1}{c|}{\begin{tabular}[c]{@{}c@{}}Software\\ Products\end{tabular}}             & \multicolumn{1}{c|}{}                                                                                                                     & \multicolumn{1}{c|}{P}                               & \multicolumn{1}{c|}{product}                                                                                                & \multicolumn{1}{c|}{\begin{tabular}[c]{@{}c@{}}Mobile\\ Software \\ Product\end{tabular}}                      & \multicolumn{1}{c|}{N.A.}                                                                                                   & \multicolumn{1}{c|}{nominal}                                                                                    & \multicolumn{1}{c|}{N.A.}                                                           & \multicolumn{1}{c|}{P1 and P2.}                                                                                                                  & N.A.                                                              \\ \hline
%\multicolumn{10}{c}{}                                                                                                                                                                                                                                                                                                                                                                                                                                                                                                                                                                                                                                                                                                                                                                                                                                                                                                                                                                                                                                                                                       \\ \hline
%\multicolumn{1}{c|}{\begin{tabular}[c]{@{}c@{}}Time of \\ implementation\end{tabular}}       & \multicolumn{1}{c|}{\multirow{2}{*}{dependent}}                                                                                           & \multicolumn{1}{c|}{\begin{equation}t\end{equation}} & \multicolumn{1}{c|}{product}                                                                                                & \multicolumn{1}{c|}{\begin{tabular}[c]{@{}c@{}}time to\\ market\end{tabular}}                                  & \multicolumn{1}{c|}{\begin{tabular}[c]{@{}c@{}}internal: time;\\  external: time to \\ market\end{tabular}}                 & \multicolumn{1}{c|}{ordinal}                                                                                    & \multicolumn{1}{c|}{Minutes} & \multicolumn{1}{c|}{\begin{tabular}[c]{@{}c@{}}From \\ 00:00:00 to \\03:00:00.\end{tabular}}                                                                                                                   & \begin{tabular}[c]{@{}c@{}}Eq. 1\end{tabular}           \\ \cline{1-1} \cline{3-10} 
%\multicolumn{1}{c|}{Faults}                                                                  & \multicolumn{1}{c|}{}                                                                                                                     & \multicolumn{1}{c|}{\begin{equation}f\end{equation}} & \multicolumn{1}{c|}{product}                                                                                                & \multicolumn{1}{c|}{\begin{tabular}[c]{@{}c@{}}number of\\ faults in\\ each\\ software\\ product\end{tabular}} & \multicolumn{1}{c|}{\begin{tabular}[c]{@{}c@{}}internal: faults;\\ external: quality\end{tabular}}                          & \multicolumn{1}{c|}{ordinal}                                                                                    & \multicolumn{1}{c|}{Integer} & \multicolumn{1}{c|}{any integer}                                                                                                                 & \begin{tabular}[c]{@{}c@{}}Eq. 2\end{tabular}           \\ \hline
%
%\end{tabular}
%}
\end{table}
%\end{landscape}

Time-to-market is the average time spent for the implementation of a software product with a specific group of variabilities of M-SPLear\allowbreak ning. With regard to the number of faults, the implemented products were tested using the concept of test cases \cite{craig02}. Thus, it was possible to quantify the mean of defects of products. Such metrics are relevant because they are directly related to time-to-market and quality of the m-learning applications.

\subsubsection{Participants}

In our study, the participants were employee volunteers from a Brazilian software development industry. All of them had, at least, one year experience with development background in Java, Microsoft .NET and/or PHP.

The reduced number of practitioners led us to apply a non-random selection. The random capacity was applied at the assignment of the development methodology and software product by participant. 

Block classification was defined by the two factors with two treatments, which were interspersed in four groups. The population was divided into four blocks by means of a draw. The balancing was applied in the tasks, which were assigned in equal numbers to a similar number of participants.

The participants were randomly separated into the following groups:

\begin{itemize}
\item \textbf{First Group:} focused on SSD with P1 and SPL with P2;

\item \textbf{Second Group:} focused on SPL with P1 and SSD with P2;

\item \textbf{Third Group:} focused on SSD with P2 and SPL with P1; and

\item \textbf{Fourth Group:} focused on SPL with P2 and SSD with P1;
\end{itemize}

\subsubsection{Objects}

Among a total of 30 features and diferent configurations, two educational software products configurations for mobile learning platform (Android) were considered for the application of the SSD and SPL methodologies which are: one for image (P1) and another for video resource (P2).

\subsubsection{Instrumentation}

The experiment was supported by the following set of instruments: (i) similar desktop computers with all necessary tools (Eclipse IDE and plugins); (ii) the consent term for the experimental study; (iii) a characterization questionnaire; (iv) use case, component and sequence UML diagrams; (v) interface messages; (vi) database model; (vii) a project base; (viii) similarities of the products; and (ix) experimental forms for SSD and SPL, randomly distributed and feedback questionnaire.

\subsubsection{Data Collection and Analysis Procedures}

The main assessment tools were the products developed based on two software specifications (P1 and P2) for mobile learning platform (Android).

The M-SPLear\allowbreak ning was designed according to the catalog of requirements (Section \ref{section2}) and with 30 features, 16 mandatory and 14 optional from m-learning applications. A specific niche of features was used for the evaluation. The variabilities related to multimedia resources enabled the creation of up to 15 different products (a represented in Figure \ref{figureMSPLFeatureModel}). P1 and P2 were specified and implemented by SSD and SPL methodologies. Figure \ref{fig:prod} shows the nuances between the products generated for the video feature.


\begin{figure*}
\centering
\includegraphics[scale=0.272]{figures/section4/prod.png}
\centering
\caption{Two Video Products (P1) Developed in the Experimental Execution with a) SPL and b) SDD.}
\label{fig:prod}
\end{figure*}


To collect the data for the analysis of time-to-market, the initial and final time of implementation process for P1 and P2, was registered individually in the experimental form to be calculated in Equation \ref{eq:1}. On the other hand, for the analysis of quality, each of 15 developed products were tested and the number of faults was collect to compare the use of SPL and SDD methodologies by means of the Equation \ref{eq:2}.

\subsubsection{Validity Evaluation}

A pilot project was developed with two pratictioners from industry, who evaluated the study instrumentation and established the duration of the training and execution sessions. The results and its participants were not used in the final execution and data analysis of the experiment.

\subsection{Execution}\label{sub:execution}

This section presents how the experimental plan (design) was defined.

\subsubsection{Sample}

The sample was composed of a total of 21 pratictioners who participated in the training session. However, 18 participants contributed in the experimental execution, due the unavailability of 3 volunteers in the execution day.

\subsubsection{Preparation}

The participants underwent a 3 day training of the essential concepts of Android development for SSD and SPL with the Eclipse IDE. The knowledge was evaluated through essays at the end of each training session. On the fourth day, the experiment was performed.

\subsubsection{Data Collection}

The steps adopted for data collection were:

\begin{enumerate}

\item the participants present of three daily 4 hour trainning sessions in an industrial environment;

\item the participants were divided into four groups by means of a draw;
\item the experimenter gave the participants a set of documents: containing UML diagram models, a dataset model and an interface message specification for each product, such as the material used in training session. Each individual was provided with a desktop computer with all requirements to develop a software product and received an experimental form to register the spending time with the development process for analysis.
\item the participant read each given document;
\item the experimenter explained the documents;
\item the participant read and clarified possible doubts about the products specifications; and
\item finally, each participant received and used two randomly drawn methodologies for the development of a requested m-learning product. For each application, participants registered the lasting of the application (start time, end and brakes). At the end of the two development tasks for the two methodologies they were asked to answer a feedback questionnaire and their opinion about the experimental execution and technologies used give.
\end{enumerate}

\subsection{Analysis}\label{sub:analysis}

As the experiment session was finished, collected data was prepared (tabulation and descriptive statistics) to be applied a statistical tests.

\subsubsection{Collected Data and Descriptive Statistics}

For each participant (``\texttt{Participant \#}'' column), we collected the following data: total time of implementation and total number of faults, identified by testing procedures, and the mean calculation. These results are shown in Table \ref{tab:resul1} and the results for each participant are plotted in box-plots of Figure \ref{fig:boxplot}. 

\begin{table}
\caption{\label{tab:resul1}SSD and SPL Collected Data and Descriptive Statistics.}
    \centering
    \scriptsize
\begin{tabular}{c|c|c|c|c}
\hline
\multirow{2}{*}{\textbf{Participant \#}} & \multicolumn{2}{c|}{\textbf{SSD}} & \multicolumn{2}{c}{\textbf{SPL}}  \\ \cline{2-5}
                                    & \textbf{Time $(t)$}   & \textbf{Faults $(f)$} & \textbf{Time $(t)$}  & \textbf{Faults $(f)$}                                  \\ \hline
1                                   & 161             & 15              & 2              & 9                                              \\ \hline
2                                   & 90              & 8               & 1              & 0                                             \\ \hline
3                                   & 105             & 4               & 11             & 0                                              \\ \hline
4                                   & 104             & 1               & 3              & 0                                             \\ \hline
5                                   & 73              & 2               & 1              & 0                                             \\ \hline
6                                   & 99              & 9               & 3              & 0                                              \\ \hline
7                                   & 165             & 12              & 10             & 0                                              \\ \hline
8                                   & 95              & 1               & 3              & 0                                              \\ \hline
9                                   & 104             & 3               & 2              & 0                                               \\ \hline
10                                  & 102             & 0               & 4              & 0                                              \\ \hline
11                                  & 61              & 0               & 2              & 0                                             \\ \hline
12                                  & 82              & 4               & 8              & 0                                              \\ \hline
13                                  & 114             & 1               & 4              & 0                                             \\ \hline
14                                  & 103             & 6               & 14             & 0                                            \\ \hline
15                                  & 111             & 2               & 2              & 0                                              \\ \hline
16                                  & 176             & 9               & 4              & 0                                              \\ \hline
17                                  & 120             & 17              & 5              & 0                                             \\ \hline
18                                  & 175             & 34              & 2              & 9                                              \\ \hline
\textbf{Mean}                       & \textbf{113.33} & \textbf{7.11}   & \textbf{4.50}  & \textbf{1.00}                 \\ \cline{1-5}
\textbf{Median}                     & \textbf{104}    & \textbf{4}      & \textbf{3}     & \textbf{0}                                      \\ \cline{1-5}
\textbf{Std. Dev.}                  & \textbf{33.98}  & \textbf{8.46}   & \textbf{3.75}  & \textbf{2.91}                             \\ \hline
\end{tabular}
\end{table}


\begin{figure}
\centering
\includegraphics[scale=0.8]{./figures/section4/boxplot.eps}
\centering
\caption{Collected Data Box-Plot from SSD and SPL faults and SSD and SPL time-to-market.}
\label{fig:boxplot}
\end{figure}

\subsubsection{Hypothesis Testing}

Based on the results obtained by the use of SSD and SPL to the development of two mobile learning products, we summarize, analyze and interpret the SSD and SPL collected data (Table \ref{tab:resul1} and Figure \ref{fig:boxplot}) by means of the Shapiro-Wilk normality test and the Mann-Whitney-Wilcoxon hypothesis test. Both tests validated the statistical power of the sample, allowing test the hypothesis.

\subsubsection{Efficiency in the Time of Implementation (R.Q.1)}

\begin{itemize}

\item \textbf{Collected Data Normality Tests:} Shapiro-Wilk \cite{shaphirowilk65} normality test was applied to SSD and SPL time and faults and the following results were obtained:\\

\textbf{\textit{SSD time (\textit{N}=18):}}

For mean value ($\mu$) of 113.33 and standard deviation value of ($\sigma$) 33.98, the time for the SSD was \textit{p} = 0.0274.

In the \textit{Shapiro-Wilk} test for a sample size \textit{(N)} 18 with 95\% of significance level ($\alpha$ = 0.05), \textit{p} = 0.0274 (0.0274 $<$ 0.05) and calculated value of \textit{W} = 0.8813 $<$ \textit{W} = 0.8970, the sample is considered non-normal.

\textbf{\textit{SPL time (\textit{N}=18):}}

For mean value ($\mu$) of 4.50 and standard deviation value of ($\sigma$) 3.75, the time for the SPL was \textit{p} = 0.0014.

For a sample size \textit{(N)} 18 with 95\% of significance level ($\alpha$ = 0.05), \textit{p} = 0.0014 (0.0014 $<$ 0.05) and calculated value of \textit{W} = 0.7978 $<$ \textit{W} = 0.8970, the sample is considered non-normal.

\item \textbf{Mann-Whitney-Wilcoxon for SSD and SPL time samples:} a rank with weights was assigned to each sample value. The weights were added and applied in Equation \ref{eq:MWW}:
\small
\begin{equation}
\begin{split}
\label{eq:MWW}
U(DM) = N_1 * N_2 + \frac{N_1*(N_1+1)}{2} - \sum_{i=1}^{n} total_{2}
\end{split}
\end{equation}
\normalsize 
Where:
\begin{itemize}
\item \textit{$U(DM)$} is the equation for each independent sample (DM);
\item \textit{$N_1$} is the size of the sample for the X methodology;
\item \textit{$N_2$} is the size of the sample for the compared methodology (Y); and
\item \textit{$total_{2}$} is the sum of the weights given for the compared methodology.
\end{itemize}

The time values calculated by Equation \ref{eq:MWW} were 326.5 for SSD and 0.00 for SPL.

Each weight matches the participants weights of development process time with SSD or SPL methodology. There are evidences that both values are different ($326.5>0$), which leads to the rejection of the null hypothesis ($H_0$) and acceptance of the alternative hypothesis ($H_{2}$).

Therefore, the answer to R.Q.1 was: SPL is more efficient than the SSD to implement software products for mobile platform. The implementations of the base project and SPL were also considered in the experiment.

The participants received the base project of the two software products to be developed with SSD or SPL. It consists of similarities of the products and was developed to reduce the experimental execution duration. It was implemented in 480 minutes (8 hours).

If we consider the time required for the implementation of each project base, plus the total development time to each participants (480 minutes), the total time would be 10680 minutes or 178 hours ($total_{time}$((subje\allowbreak cts(18) x minutes(480)) + 2040 = 10680 minutes). Taking into account the base project, the time lasting by the 18 participants was 81 minutes (1 hour and 35 minutes) and the total time would be 11361 minutes or 189 hours and 35 minutes ($total_{time}$((participants(18) x minutes(4.5)) + 10599 = 11361 minutes). 

The comparison of the values, shows the SPL development spent 621 minutes (11 hours and 35 minutes) more than SSD. However, after the SPL implementation, this approach allows the evolution and insertion of new variabilities, guaranteeing the faster generation of new products in addition to other advantages of the adoption of SPL approach.

\end{itemize}

\subsubsection{Number of faults of the created software products (R.Q.2)}

\begin{itemize}

\item \textbf{Collected Data Normality Tests:} 

\textbf{\textit{SSD faults (\textit{N}=18):}}

For a mean value ($\mu$) 7.11 and standard deviation value of ($\sigma$) 4, the fault for the SSD was \textit{p} = 0.0006 for the \textit{Shapiro-Wilk} normality test.

For a sample size \textit{(N)} 18 with 95\% significance level ($\alpha$ = 0.05), \textit{p} = 0.0006 (0.0006 $<$ 0.05) and value of \textit{W} = 0.7740 $<$ \textit{W} = 0.8970, the sample is considered non-normal.

\textbf{\textit{SPL faults (\textit{N}=18):}}

For a mean value ($\mu$) 1.00 and standard deviation value of ($\sigma$) 0, the fault for the SPL was \textit{p} = 0.00000007 for the \textit{Shapiro-Wilk} normality test.

For a sample size \textit{(N)} 18 with 95\% of significance level ($\alpha$ = 0.05), \textit{p} = 0.00000007 (0.00000007 $<$ 0.05) and value of \textit{W} = 0.3730 $<$ \textit{W} = 0.8970, the sample is considered non-normal.

\item \textbf{Mann-Whitney-Wilcoxon for SSD and SPL faults samples:} the number of faults calculated for SSD by Equation \ref{eq:MWW} was 282, whereas and for SPL, it was 42.

Each weight matches participants development project faults with SSD or SPL methodology. There are evidences that both values are different ($282>42$), which leads to the rejection of the null hypothesis ($H_0$) and acceptance of the alternative hypothesis ($H_{1}$).

According to the result from the Mann-Whitney-Wilcoxon, the answer for R.Q.2 was obtained: it means that the SSD is prone to showed more faults in the software products developed than the SPL.

\end{itemize}

\subsection{Interpretation and Discussion}\label{sub:interpretation}

Data collected from the SSD and SPL application was analyzed and interpreted. The results are summarized in Table \ref{tab:resul_s}.

\begin{table}
\caption{\label{tab:resul_s}SSD and SPL Normality and Statistical Tests Results.}
\centering
\includegraphics[scale=0.35]{figures/section4/MSPLExpSummary.png}

%\scriptsize
%\resizebox{0.87\textwidth}{!}{\begin{minipage}{\textwidth}
%\begin{tabular}{ccc}
%\hline
%\multicolumn{1}{c|}{\textbf{Element}}                                      & \multicolumn{1}{c|}{\textbf{SSD}}                          & \textbf{SPL}                                 \\ \hline
%\multicolumn{1}{c|}{\multirow{2}{*}{\textbf{Selection of Participants}}}       & \multicolumn{2}{c}{18 practitioners.}                                                                     \\ \cline{2-3} 
%\multicolumn{1}{c|}{}                                                      & \multicolumn{1}{c|}{N(SSD)=18}                             & N(SPL)=18                                    \\ \hline
%\multicolumn{3}{c}{\textbf{Time to Implementation}}                                                                                                                                            \\ \hline
%\multicolumn{1}{c|}{\textbf{Mean}}                                         & \multicolumn{1}{c|}{113.33}                                & 4.5                                          \\ \hline 
%\multicolumn{1}{c|}{\multirow{2}{*}{\textbf{Shapiro-Wilk Normality Test}}} & \multicolumn{1}{c|}{p = 0.0274 (0.0274 \textless 0.05)}    & p = 0.0014 (0.0014 \textless 0.05)           \\
%\multicolumn{1}{c|}{}                                                      & \multicolumn{1}{c|}{Non-normal.}                           & Non-normal.                                  \\ \hline
%\multicolumn{1}{c|}{\multirow{2}{*}{\textbf{Statistical Test}}}            & \multicolumn{2}{c}{Mann-Whitney-Wilcoxon}                                                                 \\
%\multicolumn{1}{c|}{}                                                      & \multicolumn{2}{c}{(326.5 \textgreater 0) Evidence of statistical differences among times.}            \\ \hline
%\multicolumn{1}{c|}{\textbf{Result}}                                       & \multicolumn{2}{c}{R.Q.1 $H_2$: SPL has a smaller time-to-market than SSD.}                                                      \\ \hline
%\multicolumn{3}{c}{\textbf{Number of Faults}}                                                                                                                                \\ \hline
%\multicolumn{1}{c|}{\textbf{Mean}}                                         & \multicolumn{1}{c|}{7.11}                                  & 1                                            \\ \hline
%\multicolumn{1}{c|}{\textbf{Shapiro-Wilk Normality Test}}                  & \multicolumn{1}{c|}{p = 0.0006 (0.0006 \textless 0.05)}    & p = 0.00000007 (0.00000007 \textless 0.05)   \\ \hline
%\multicolumn{1}{c|}{\multirow{2}{*}{\textbf{Statistical Test}}}            & \multicolumn{2}{c}{Mann-Whitney-Wilcoxon}                                                                 \\
%\multicolumn{1}{c|}{}                                                      & \multicolumn{2}{c}{(282 \textgreater 42) Evidence of statistical differences among number of faults.} \\ \hline
%\multicolumn{1}{c|}{\textbf{Result}}                                       & \multicolumn{2}{c}{R.Q.2 $H_1$: SPL has a smaller number of faults in the products than SSD.}                                    
%\end{tabular}
%\end{minipage}}
\end{table}

In terms of time-to-market, the statistical difference showed by Mann-Whitney-Wilcoxon test provides evidence that SPL (i.e., M-SPLear\allowbreak ning) was more efficient than SSD in the development of P1 and P2 m-learning products, therefore, R.Q.1 has been answered.

Regarding number of faults, the statistical difference presented by Mann-Whitney-Wilcoxon test provides evidence that SSD showed more faults than SPL in the development of P1 and P2 m-learning products, therefore, R.Q.2 has been answered.

According to the results of the Mann-Whitney-Wilcoxon test, both R.Q.1 and R.Q.2 null hypotheses can be rejected.

\subsubsection{Threats to Validity}\label{sec:threats}

This section addresses the actions taken to play directly against threats of this experiment, according to the Conceptual Model of Anderlin Neto and Conte \cite{neto13}.

\textbf{Internal Validity:}

\begin{itemize}
\item \textbf{Differences among participants:} as we selected participants with different experience levels, variations in this skills were reduced during the training sessions. The assessments conducted at in the end of each day of training demonstrated the level of knowledge in the content used in the experimental execution and guaranteed the reduction in variations in the participants skills. Even knowing that a more homogeneous sample reduce the subjects representativeness, it was decided to apply the training to reduce the heterogeneity of participants, that could threat the conclusion validity.

\item \textbf{Fatigue effects:} on average, the experiment lasted 180 minutes. Fatigue was not considered relevant since the participants could leave the room for a quick break. They were warned to not communicate during the breaks and, to guarantee, an human observer supervised them. Periods of absence were registered and disregarded in the time analysed.



\item \textbf{Influence among participants:} the participants performed the experiment under the supervision of a human observer, so that a possible influence of communication among them could be mitigated. As participants behaves differently when being observed, training sessions allowed the adaptation the participants to the environment, reducing this threat.

\item \textbf{Trainning Sessions:} in the training sessions, explanations were given for every participant. This action was taken to avoid possible biases, and allow that every training member solve all your questions.
\end{itemize}

\textbf{External Validity:}

\begin{itemize}

\item \textbf{Instrumentation:} m-learning products and other instruments were tested in the pilot project and were considered significant for the analysis of time-to-market and number of faults.

\item \textbf{Participants:} more experiments that considering different metrics with industry practitioners must be conducted for the identification of other relevant factors related to the adoption of M-SPLear\allowbreak ning.

\end{itemize}

\begin{itemize}

\item \textbf{Construction Validity:} independent variables were tested in the pilot project to guarantee their validity.

\item \textbf{Conclusion Validity:} since the number of participants is reduced, mainly by the availability of practitioners in the industry, the sample size must be increased in prospective replications of the experiment. Our results are considered indicators and are not conclusive, although the lack of experimental executions in an industrial environment, even with small samples, is important for the evaluation of the time-to-market and quality for both SSD and SPL.

\end{itemize}


\section{Lessons Learned}\label{section5}

During the execution of the activities documented in this paper, the authors identified some situations in which works related to the concepts involved can benefit. As lessons learned we highlight:

\begin{itemize}
\item \textbf{Domain Characteristics:} domain analysis can be considered one of the most important activities for the creation of an SPL. The evolution of the catalog of requirements proposed by Duarte Filho and Barbosa \cite{filho13} significantly contributed in terms of domain knowledge and supported the adoption of the proactive model for the development of M-SPLear\allowbreak ning.
    
\item \textbf{Variability Management:} the use of the SMarty notation helps the identification of the variation points during the design of M-SPLear\allowbreak ning, ensuring greater cohesion for the implementation of the SPL components. It also contributes to the assimilation of the SPL concept. However, those involved must be trained so that the elements of notation can be used consistently.
    
\item \textbf{SPL Architecture:} SPL needs mechanisms to allow a transparent and easy manner to reflect all updates in the core assets in the architecture, and vice versa. In other words, changes to the core assets require more efforts to maintain architectural integrity. Additional tools and analysis must be done to guarantee that all changes, in the architecture or in the components, are reflecting all features and behaviors that the line has until that moment.
    
\item \textbf{SPL Development:} considering M-SPLear\allowbreak ning, the implementation of something generic and customizable is significantly different from a static approach. Therefore, developing features in an SPL requires a greater effort, which is justified by the subsequent gains of reuse.

On another hand, the singular development methodology difficults the maintenance of the products, without any rigorous development process defined. As all developers tend to program in their own way, give support for each product developed, in most cases, could take most time and cost more. If an SPL practice is adopted, being rigorously followed, the cost and time to give support tend to decrease.
    
\item \textbf{Experimental Evaluation:} researches show that test executions in SPL are scarce and need to be evaluated and validated \cite{engstrom11}. Thus, the authors decided to apply the test cases in the products generated by the SPL, enabling an interesting comparison with an alternative methodology of development.

The experimental evaluation provided relevant results for the adoption of M-SPLear\allowbreak ning. The choice for active participants in the industry contributed to the reduction of the training session. However, experience and understanding the concepts by the participants is always difficult to measure.

Moreover, the use of tests techniques in the traditional development process and new ways to test quickly products from an SPL require more attention and research. One benefit in the adoption of SPL is about the improve of quality, since the products and their components are tested in several instantiations, leading to a quickly fix for the final client. Although the use of the components in a large number of products and, for consequently, a bigger number of target users for SPL, the way that the components and products were tested could be improved to be done in more aligned manner with the SPL specificities. Literature reflects more concern to test the SPL architecture and fewer preoccupation to test the final components and products.
\end{itemize}

\section{Related Work} \label{section6}

Our work encompass three main perspectives in industry environments: (i) m-learning applications, (ii) SPL with its benefits through variability management and (iii) Experimental Software Engineering. Adopting SPL as methodology to develop m-learning applications allowed us to collect positive evidences in a real industry environment about two measurable SPL benefits: quality of products and time-to-market, when compared experimentally with a singular software development process without a variability management approach to support the developers. Thus, our results come to complement the experiences related in other researches for these three perspectives.

In mobile applications development domain, Gamez et al. \cite{gamez14} proposed a self-adaptation of mobile systems with dynamic SPL. The management of variabilities is achieved using the Common Variability Language (CVL). Marinho et al. \cite{marinho10} proposed an architecture for nested SPL in the domain of mobile and context-aware applications. However, the authors did not specify how to improve the management of variabilities. Bezerra et al. \cite{bezerra09} conducted a systematic review of SPL applied to mobile middlewares, but only six studies were significant for the review. According to the authors, the few results obtained highlight the need of more research in the area. 

These evidences led us to align the necessity to grow the number of m-learning applications -- which presents a still incipient exploration in literature -- using a reused-based approach. In our case, SPL methodology with SMarty to manage the variabilities. Analysing the literature, as for m-learning development, there is also a lack in researches about the use of SPL and variability management as a way to improve quality and time-to-market in industry. Chen and Babar \cite{chen11}, in a systematic literature review, concluded the status of evaluation of variability management approaches in SPL engineering was quite unsatisfactory. 

Jaring and Bosch \cite{jaring02} presents a case study in a Dutch-based company, Rohill Technologies BV -- a system development for, mainly, professional mobile communication infrastructures. They discussed and analyzed the need for handling variability in a more explicit manner, discussing the SPL and a method to represent and normalize variabilities. Some issues were highlighted, as limited insight into the consequences of selecting a particular variability mechanism for variability identification; a notation format to describe variability is not available and dependencies between variability and features are not made explicit to variability dependencies.

Hubaux et al \cite{hubaux10} combined variability representation and industrial case studies evaluations. They developed a Textual Variability Language (TVL) combining graphical and textual notations and conducted an evaluation through a quantitative and qualitative analysis in four cases from different companies, sizes and domains. Positive benefits for TVL were identified, as efficiency gained in terms of model comprehension, design and learning curve for the notation. Some limitations were presented as well. Eriksson et al \cite{eriksson09} also presented a language, but to manage requirements specifications for SPL. Using a qualitative evaluation -- document examination, participant observation and semi-structure interviews -- they compared the clone-an-own reuse with the proposed approach, which must be use together with a previously developed method to manage product line use cases models. Six variables were analyzed: adoption effort, expressiveness, scalability, ROI, risks and reuse infrastructure. The ROI variable presented negative results, the others presented positive evidences for the proposed language. 

Still in the context of SPL evaluation in industry, but with focus on quality through software testing, Ardis et al \cite{ardis00} used the Family-Oriented Abstraction, Specification and Translation (FAST) approach as a development process for an SPL in a case study, covering all aspects of domain analysis with tests. According to the authors, test process in an SPL presents significant challenges. Thus, they presented, for each case study, three test strategies suitable for general use in SPL testing: (i) testing common code thoroughly; (ii) exploiting common aspects of variable code; and (iii) utilize scenarios from the commonality analysis. 

Gacek et al \cite{gacek01} present a study case in the adoption of SPL in a small company called Market Maker. An holistic view of the challenges and changes in business was discussed, especially the automatization of tests in their developed components. Firstly, each single component was tested and, then, the entire system at runtime, using a special code inserted in each component. If an error occurred, the inserted test code identifies automatically which component misbehave. The testing process was conducted only for components that are present in the instantiated product, in other words, components are not tested in their full flexibility, with respect to all possible instantiations in the family.

Batory et al \cite{batory02} define a Product Line Architecture (PLA) with a Domain-Specific Language (DSL) to redesign an extensible command-and-control simulator for army fire support. In the case study, they collected preliminary results that PLA with DSL produce a more flexible way to implements the simulator and reduced the program complexity,if compared with the same simulator, with pure Java implementation. The complexity of the code was compared based in the number of methods, number of line codes and tokens/symbols for both the adopted approach and the Java implementation.

Finally, observing the related works we noticed that, in spite of all of them were applied in industry, only few of them compare methodologies, as SPL and singular development, highlighting which of them bring more quality and reduces time-to-market, when supported by a variability management approach. Additionally, there are several opportunities and open issues regarding the management of variabilities and experimental evaluation, particularly in the m-learning domain. Such needs of research in the area has motivated our work.

\section{Conclusions and Future Work}\label{section7}

%Atualmente, de acordo com a International Telecommunication Union (ITU), a quantidade de assinaturas móveis em todo o mundo aproxima-se do número de pessoas na terra. Nesse contexto, a plataforma Android mostrou-se mais relevante no domínio das aplicações m-learning, devido a sua quantidade de usuários potenciais. Com isso, este trabalho propõe uma LPS cujo principal objetivo é proporcionar maior reúso e padronização para esse domínio de aplicações.
%Nowadays, according to the International Telecommunication Union (ITU) \cite{itu14}, the number of mobile subscriptions around the world approaches the number of people on Earth. In this context, the Android platform was more relevant in the field of m-learning applications, due to its amount of potential users. Thus, 
This work proposes an SPL with the main objective is to provide greater reuse and standardization for producing m-learning applications.
%For this, we discussed how the variability management can improve the development of software products, particularly in the context of m-learning applications. We described M-SPLear\allowbreak ning, which supports the development of customizable m-learning applications according to the basics of SPLs.
%Nesse cenário, o tempo necessário para o desenvolvimento dessas aplicações é essencial para que elas cheguem mais rapidamente a seus usuários finais. Desta forma, o estudo do conceito de LPS convergiu com a necessidade em questão e resultou na concepção da M-SPLear\allowbreak ning.
In this scenario, the time required for the development of such applications is essential to come more quickly to their end users. Thus, the study of the concept of SPL converged to the need in question and resulted in the conception of M-SPLear\allowbreak ning.

To support our ideas, we experimentally evaluated the use of M-SPLear\allowbreak ning with respect to the singular software development. The obtained results were significant for the reuse approach, showing a reduction on time-to-market and a better quality in terms of faults when considering the software products developed with the support of variabilities. 

It is also important to highlight that the SMarty approach was crucial to the development of M-SPLear\allowbreak ning, providing cost savings and better quality to the software products developed.

%Como trabalhos futuros, pretende-se evoluir a M-SPLear\allowbreak ning com base nos insumos fornecidos a partir de sua avaliação empírica. Nesse sentido, ainda existe uma quantidade significativa de informações que podem induzir a novas linhas de pesquisa e avaliações experimentais.
As future work, we intend to evolve M-SPLear\allowbreak ning based on the inputs provided by the empirical evaluation performed. In this sense, there is still a significant amount of information that can lead to new research lines and experimental evaluations.

%Além disso, para que a M-SPLear\allowbreak ning possa ser avaliada em sua totalidade, todas as features elicitadas pelo catálogo de requisitos devem ser devidamente implementadas. Desta forma, um estudo empírico envolvendo a LPS propriamente dita pode ser conduzido, porque nosso experimento aferiu apenas os produtos da M-SPLear\allowbreak ning.
Moreover, to a whole evaluation of M-SPLear\allowbreak ning, all features elicited by catalog requirements must be properly implemented. Therefore, an empirical study involving the LPS itself can be conducted, because our experiment measured only M-SPLear\allowbreak ning products.

%Considerando o domínio explorado, a plataforma Android vem recebendo constantes contribuições em sua estratégia de desenvolvimento. Com isso, o aperfeiçoamento da M-SPLear\allowbreak ning deve sempre considerar a avaliação de novas ferramentas, fazendo com que a LPS evolua de acordo com as tendências do mercado.
%Considering the explored domain, the Android platform has received ongoing contributions to its development strategy. Thus, the improvement of M-SPLear\allowbreak ning must always consider the evaluation of new tools, causing the LPS evolve according to market trends.

%Para avaliação das evoluções da M-SPLear\allowbreak ning novos experimentos devem ser conduzidos, explorando vertentes relevantes para contexto educacional. Nesse sentido, avaliar os produtos gerados considerando variáveis como usabilidade e efetividade devem proporcionar resultados expressivos para este estudo. Além disso, as aplicações móveis resultantes podem ser aplicadas  em cenários reais de ensino e aprendizagem, com o objetivo de avaliar a M-SPLear\allowbreak ning aplicada ao seu domínio de usuários.
%evolutions of 
To evaluate the M-SPLear\allowbreak ning new experiments should be conducted, exploring relevant  aspects to educational context. In this sense, evaluate the products considering variables such as usability and effectiveness should provide significant results for this study. In addition, the resulting mobile applications can be applied in real scenarios of teaching and learning, in order to evaluate the M-SPLear\allowbreak ning applied to your potential users.

%Outra perspectiva interessante está relacionada à investigação das outras estratégias de adoção propostas por. O modelo extrativo poderia ser aplicado a produtos externos a M-SPLear\allowbreak ning, com o objetivo de expandir suas features. Além disso, o modelo reativo poderia ser estudado como uma alternativa para evoluções futuras da LPS proposta.
We also intend to investigate the use of other adoption models \cite{krueger02}. For instance, the extractive model can be applied in similar products to those generated by M-SPLear\allowbreak ning aiming at increasing the validity of similarities and variabilities specified. The reactive model can be investigated as an alternative to the evolution of the proposed SPL as well.

\input{sections/section8}

\begin{thebibliography}{99}

\harvarditem{Chen}{2011}{chen11} Chen~L, Babar~MA. 2011. A systematic review of evaluation of variability management approaches in software product lines. Information and Software Technology.

\harvarditem{Capilla et al}{2013}{capilla13} Capilla~R, Bosch~J, Kang~KC. 2013. Systems and Software Variability Management. Springer.

\harvarditem{B\"{o}ckle and van der Linden}{2005}{bockle05} B\"{o}ckle~G, van der Linden~FJ. 2005. Software product line engineering: foundations, principles and techniques. Edited by Klaus Pohl. Springer Science \& Business Media.

\harvarditem{van der Linden et al}{2007}{vanderlinden07} van der Linden~FJ, Schmid~K, Rommes~E. 2007. Software product lines in action: the best industrial practice in product line engineering. Springer Science \& Business Media.

\harvarditem{West and Vosloo}{2012}{west12} West~M, Vosloo~S. 2012. Mobile Learning for Teachers: Global Themes. UNESCO.

\harvarditem{Kukulska-Hulme and Traxler}{2005}{kukulska05} Kukulska-Hulme~A, Traxler~J. 2005. Mobile Learning: a Handbook for Educators and Trainers. Routledge.

\harvarditem{Bezerra et al}{2009}{bezerra09} Bezerra~YM, Pereira~TAB, da Silveira~GE. 2009. A Systematic Review of Software Product Lines Applied to Mobile Middleware. Information Technology: New Generations (ITNG).

\harvarditem{Kinshuk et al}{2003}{kinshuk03} Kinshuk~SJ, Sutinen~E, Goh~T. 2003. Mobile technologies in support of distance learning. Asian Journal of Distance Education.

\harvarditem{Wexler et al}{2008}{wexler08} Wexler~S, Brown~J, Metcalf~D, Rogers~D, Wagner~E. 2008. Mobile Learning: What it is, why it matters, and how to incorporate it into your learning strategy. Guild Research.

\harvarditem{FalvoJr et al}{2014a}{falvojr14a} FalvoJr~V, Duarte Filho~NF, OliveiraJr~E, Barbosa~EF. 2014. Towards the Establishment of a Software Product Line for Mobile Learning Applications. International Conference on Software Engineering and Knowledge Engineering (SEKE).

\harvarditem{FalvoJr et al}{2014b}{falvojr14b} FalvoJr~V, Duarte Filho~NF, OliveiraJr~E, Barbosa~EF. 2014. A Contribution to the Adoption of Software Product Lines in the Development of Mobile Learning Applications. Annual Frontiers In Education (FIE) Conference.

\harvarditem{OliveiraJr et al}{2010}{oliveirajr10} OliveiraJr E, Gimenes IMS, Maldonado JC. 2010. Systematic Management of Variability in UML-based Software Product Lines. Journal Universal Computer Science.

\harvarditem{Bosch}{2001}{bosch01} Bosch~J. 2001. Software product lines: organizational alternatives. International Conference on Software Engineering.

\harvarditem{Marcolino et al}{2013}{marcolino13} Marcolino~A, OliveiraJr~E, Gimenes~IMS, Maldonado~JC. 2013. Towards the Effectiveness of a Variability Management Approach at Use Case Level. International Conference on Software Engineering and Knowledge Engineering (SEKE).

\harvarditem{Marcolino et al}{2014a}{marcolino14a} Marcolino~A., OliveiraJr~E, Gimenes~IMS, Barbosa~EF. 2014. Empirically Based Evolution of a Variability Management Approach at UML Class Level. Computer Software and Applications Conference (COMPSAC).

\harvarditem{Marcolino et al}{2014b}{marcolino14b} Marcolino~A., OliveiraJr~E, Gimenes~IMS. 2014. Towards the Effectiveness of the SMarty Approach for Variability Management at Sequence Diagram Level. International Conference on Enterprise Information Systems (ICEIS).

\harvarditem{Bera et al}{2015}{bera15} Bera~MHG, OliveiraJr~E, Colanzi~TE. 2015. Evidence-based SMarty Support for Variability Identification and Representation in Component Models. International Conference on Enterprise Information Systems (ICEIS).

\harvarditem{Duarte~Filho and Barbosa}{2013}{filho13} Duarte~Filho~NF, Barbosa~EF. 2013. A requirements catalog for mobile learning environments. Annual ACM Symposium on Applied Computing.

\harvarditem{Krueger}{2002}{krueger02} Krueger~C. 2002. Easing the transition to software mass customization. Software Product-Family Engineering. Springer Berlin Heidelberg.

\harvarditem{Kang et al}{1990}{kang90} Kang~KC, Cohen~SG, Hess~JA, Novak~WE, Peterson~AS. 1990. Feature-oriented domain analysis (FODA) feasibility study. Software Engineering Institute, Carnegie Mellon University.

\harvarditem{Llamas et al}{2014}{llamas14} Llamas~MSR, Reith~R, Shirer~M. 2014. Android and iOS Continue to Dominate the Worldwide Smartphone Market with Android Shipments Just Shy of 800 Million in 2013. According to IDC (Online).

\harvarditem{Van Gurp et al}{2001}{vangurp01} Van Gurp~J, Bosch~J, Svahnberg~M. 2001. On the notion of variability in software product lines. Working IEEE/IFIP Conference on Software Architecture.

\harvarditem{Fielding}{2000}{fielding00} Fielding~RT. 2000. Architectural styles and the design of network-based software architectures. University of California.

\harvarditem{Marinho et al}{2010}{marinho10} Marinho~FG, Costa~AL, Lima~FF, Neto~JBB, Rocha~L, Dantas~VLL, Andrade~RMC, Teixeira~E, Werner~C. 2010. An architecture proposal for nested software product lines in the domain of mobile and context-aware applications. Software Components, Architectures and Reuse (SBCARS).

\harvarditem{Nascimento et al}{2011}{nascimento11} Nascimento~AS, Rubira~CMF and Lee~J. 2011. An SPL approach for adaptive fault tolerance in SOA. International Software Product Line Conference.

\harvarditem{Jedlitschka and Pfahl}{2005}{jedlitschka07} Jedlitschka~A, Pfahl~D 2005. Reporting guidelines for controlled experiments in software engineering. In Empirical Software Engineering. IEEE Software.

\harvarditem{Wohlin et al}{2012}{wohlin12} Wohlin~C, Runeson~P, Host~M, Ohlsson~MC, Regnell~B, Wesslen~A. 2012. Experimentation in software engineering. Springer Science \& Business Media.

\harvarditem{Juristo and Moreno}{2010}{juristo10} Juristo~N and Moreno~ AM. 2010. Basics of Software Engineering Experimentation (1st ed.). Springer Publishing Company, Incorporated.

\harvarditem{Craig and Jaskiel}{2002}{craig02} Craig~RD, Jaskiel~SP. 2002. Systematic software testing. Artech House.

\harvarditem{Shaphiro and Wilk}{1965}{shaphirowilk65} Shaphiro~SS, Wilk~MB. 1965. An analysis of variance test for normality. Biometrika.

\harvarditem{Neto and Conte}{2013}{neto13} Neto~AA, Conte~T. 2013. A conceptual model to address threats to validity in controlled experiments. International Conference on Evaluation and Assessment in Software Engineering. ACM.

\harvarditem{Engstr\"{o}m and Runeson}{2011}{engstrom11} Engstr\"{o}m~E, Runeson~P. 2011. Software product line testing -- A systematic mapping study. Information and Software Technology.

\harvarditem{Gamez et al}{2014}{gamez14} Gamez~N, Fuentes~L, Troya~J. 2014. Self-adaptation of mobile systems with dynamic software product lines. IEEE Software.

\harvarditem{ITU}{2014}{itu14} ITU. 2014. The world in 2014: Ict facts and figures. Information and Communication Technologies (ICT).

\harvarditem{Jaring and Bosch}{2002}{jaring02} Jaring M and Bosch J. 2002. Representing Variability in Software Product Lines: A Case Study, in Software Product Lines : Second International Conference, SPLC 2, San Diego, CA, USA, August 19-22, 2002. Proceedings, vol. LNCS 2379, G. Chastek, Ed.: Springer Berlin / Heidelberg, pp. 219-245.

\harvarditem{Hubaux et al}{2010}{hubaux10} Hubaux A, Boucher Q, Hartmann H, Michel R, and Heymans P. 2010. Evaluating a Textual Feature Modelling Language: Four Industrial Case Studies, presented at 3rd International Conference on Software Language Engineering (SLE'10), Eindhoven, Netherlands.

\harvarditem{Eriksson et al}{2009}{eriksson09} Eriksson M, Börstler J, and Kjell B. 2009. Managing requirements specifications for product lines - An approach and industry case study, Journal of Systems and Software, vol. 82, pp. 435-447.

\harvarditem{Ardis et al}{2000}{ardis00} Ardis M, Daley N, Hoffman D, Siy H, and Weiss D. 2000. Software product lines: a case study, Software: Practice and Experience, vol. 30, pp. 825-847.

\harvarditem{Gacek et al}{2001}{gacek01} Gacek C, Knauber P, Schmid K, and Clements P. 2001. Successful Software Product Line Development in a Small Organization - A Case Study, IESE-Report No. 013.01/E, Fraunhofer IESE.

\harvarditem{Batory et al}{2002}{batory02} Batory D, Johnson C, MacDonald B, and von Heeder D. 2002. Achieving Extensibility Through Product-Lines and Domain-Specific Languages: A Case Study, ACM Transactions on Software Engineering and Methodology, vol. 11, pp. 191-214.

\end{thebibliography}

%\section*{Author Biographies}
%\leavevmode
%
%\vbox{%
%\begin{wrapfigure}{l}{80pt}
%{\vspace*{15pt}\fbox{insert photo}\vspace*{100pt}}%
%\end{wrapfigure}
%\noindent\small 
%{\bf Venilton FalvoJr} TODO \vadjust{\vspace{40pt}}}
%
%\vspace{15pt}
%\vbox{%
%\begin{wrapfigure}{l}{80pt}
%{\vspace*{15pt}\fbox{insert photo}\vspace*{100pt}}%
%\end{wrapfigure}
%\noindent\small {\bf Anderson~Marcolino} TODO \vadjust{\vspace{40pt}}}
%
%\vspace{15pt}
%\vbox{%
%\begin{wrapfigure}{l}{80pt}
%{\vspace*{15pt}\fbox{insert photo}\vspace*{100pt}}%
%\end{wrapfigure}
%\noindent\small 
%{\bf Nemesio~Duarte~Filho} TODO \vadjust{\vspace{40pt}}}
%
%\vspace{15pt}
%\vbox{%
%\begin{wrapfigure}{l}{80pt}
%{\vspace*{15pt}\fbox{insert photo}\vspace*{100pt}}%
%\end{wrapfigure}
%\noindent\small 
%{\bf Edson~OliveiraJr} TODO \vadjust{\vspace{40pt}}}
%
%\vspace{15pt}
%\vbox{%
%\begin{wrapfigure}{l}{80pt}
%{\vspace*{15pt}\fbox{insert photo}\vspace*{100pt}}%
%\end{wrapfigure}
%\noindent\small 
%{\bf Ellen~Barbosa} TODO \vadjust{\vspace{40pt}}}
%
%\correspond{Ellen Barbosa, Institute of Science and Computational Mathematics (ICMC-USP), Avenida Trabalhador S\~ao-Carlense, 400, S\~ao Paulo, Brazil. E-mail: francine@icmc.usp.br. Phone: +55 (16) 3373-8177. Fax: +55 (16) 3373-8888.}
\label{lastpage}
\end{document}
