\section{Lessons Learned}\label{section5}

During the execution of the activities documented in this paper, the authors identified some situations in which works related to the concepts involved can benefit. As lessons learned we highlight:

\begin{itemize}
\item \textbf{Domain Characteristics:} domain analysis can be considered one of the most important activities for the creation of an SPL. The evolution of the catalog of requirements proposed by Duarte Filho and Barbosa \cite{filho13} significantly contributed in terms of domain knowledge and supported the adoption of the proactive model for the development of M-SPLear\allowbreak ning.
    
\item \textbf{Variability Management:} the use of the SMarty notation helps the identification of the variation points during the design of M-SPLear\allowbreak ning, ensuring greater cohesion for the implementation of the SPL components. It also contributes to the assimilation of the SPL concept. However, those involved must be trained so that the elements of notation can be used consistently.
    
\item \textbf{SPL Architecture:} SPL needs mechanisms to allow a transparent and easy manner to reflect all updates in the core assets in the architecture, and vice versa. In other words, changes to the core assets require more efforts to maintain architectural integrity. Additional tools and analysis must be done to guarantee that all changes, in the architecture or in the components, are reflecting all features and behaviors that the line has until that moment.
    
\item \textbf{SPL Development:} considering M-SPLear\allowbreak ning, the implementation of something generic and customizable is significantly different from a static approach. Therefore, developing features in an SPL requires a greater effort, which is justified by the subsequent gains of reuse.

On another hand, the singular development methodology difficults the maintenance of the products, without any rigorous development process defined. As all developers tend to program in their own way, give support for each product developed, in most cases, could take most time and cost more. If an SPL practice is adopted, being rigorously followed, the cost and time to give support tend to decrease.
    
\item \textbf{Experimental Evaluation:} researches show that test executions in SPL are scarce and need to be evaluated and validated \cite{engstrom11}. Thus, the authors decided to apply the test cases in the products generated by the SPL, enabling an interesting comparison with an alternative methodology of development.

The experimental evaluation provided relevant results for the adoption of M-SPLear\allowbreak ning. The choice for active participants in the industry contributed to the reduction of the training session. However, experience and understanding the concepts by the participants is always difficult to measure.

Moreover, the use of tests techniques in the traditional development process and new ways to test quickly products from an SPL require more attention and research. One benefit in the adoption of SPL is about the improve of quality, since the products and their components are tested in several instantiations, leading to a quickly fix for the final client. Although the use of the components in a large number of products and, for consequently, a bigger number of target users for SPL, the way that the components and products were tested could be improved to be done in more aligned manner with the SPL specificities. Literature reflects more concern to test the SPL architecture and fewer preoccupation to test the final components and products.
\end{itemize}