\section{Conclusions and Future Work}\label{section7}

%Atualmente, de acordo com a International Telecommunication Union (ITU), a quantidade de assinaturas móveis em todo o mundo aproxima-se do número de pessoas na terra. Nesse contexto, a plataforma Android mostrou-se mais relevante no domínio das aplicações m-learning, devido a sua quantidade de usuários potenciais. Com isso, este trabalho propõe uma LPS cujo principal objetivo é proporcionar maior reúso e padronização para esse domínio de aplicações.
%Nowadays, according to the International Telecommunication Union (ITU) \cite{itu14}, the number of mobile subscriptions around the world approaches the number of people on Earth. In this context, the Android platform was more relevant in the field of m-learning applications, due to its amount of potential users. Thus, 
This work proposes an SPL with the main objective is to provide greater reuse and standardization for producing m-learning applications.
%For this, we discussed how the variability management can improve the development of software products, particularly in the context of m-learning applications. We described M-SPLear\allowbreak ning, which supports the development of customizable m-learning applications according to the basics of SPLs.
%Nesse cenário, o tempo necessário para o desenvolvimento dessas aplicações é essencial para que elas cheguem mais rapidamente a seus usuários finais. Desta forma, o estudo do conceito de LPS convergiu com a necessidade em questão e resultou na concepção da M-SPLear\allowbreak ning.
In this scenario, the time required for the development of such applications is essential to come more quickly to their end users. Thus, the study of the concept of SPL converged to the need in question and resulted in the conception of M-SPLear\allowbreak ning.

To support our ideas, we experimentally evaluated the use of M-SPLear\allowbreak ning with respect to the singular software development. The obtained results were significant for the reuse approach, showing a reduction on time-to-market and a better quality in terms of faults when considering the software products developed with the support of variabilities. 

It is also important to highlight that the SMarty approach was crucial to the development of M-SPLear\allowbreak ning, providing cost savings and better quality to the software products developed.

%Como trabalhos futuros, pretende-se evoluir a M-SPLear\allowbreak ning com base nos insumos fornecidos a partir de sua avaliação empírica. Nesse sentido, ainda existe uma quantidade significativa de informações que podem induzir a novas linhas de pesquisa e avaliações experimentais.
As future work, we intend to evolve M-SPLear\allowbreak ning based on the inputs provided by the empirical evaluation performed. In this sense, there is still a significant amount of information that can lead to new research lines and experimental evaluations.

%Além disso, para que a M-SPLear\allowbreak ning possa ser avaliada em sua totalidade, todas as features elicitadas pelo catálogo de requisitos devem ser devidamente implementadas. Desta forma, um estudo empírico envolvendo a LPS propriamente dita pode ser conduzido, porque nosso experimento aferiu apenas os produtos da M-SPLear\allowbreak ning.
Moreover, to a whole evaluation of M-SPLear\allowbreak ning, all features elicited by catalog requirements must be properly implemented. Therefore, an empirical study involving the LPS itself can be conducted, because our experiment measured only M-SPLear\allowbreak ning products.

%Considerando o domínio explorado, a plataforma Android vem recebendo constantes contribuições em sua estratégia de desenvolvimento. Com isso, o aperfeiçoamento da M-SPLear\allowbreak ning deve sempre considerar a avaliação de novas ferramentas, fazendo com que a LPS evolua de acordo com as tendências do mercado.
%Considering the explored domain, the Android platform has received ongoing contributions to its development strategy. Thus, the improvement of M-SPLear\allowbreak ning must always consider the evaluation of new tools, causing the LPS evolve according to market trends.

%Para avaliação das evoluções da M-SPLear\allowbreak ning novos experimentos devem ser conduzidos, explorando vertentes relevantes para contexto educacional. Nesse sentido, avaliar os produtos gerados considerando variáveis como usabilidade e efetividade devem proporcionar resultados expressivos para este estudo. Além disso, as aplicações móveis resultantes podem ser aplicadas  em cenários reais de ensino e aprendizagem, com o objetivo de avaliar a M-SPLear\allowbreak ning aplicada ao seu domínio de usuários.
%evolutions of 
To evaluate the M-SPLear\allowbreak ning new experiments should be conducted, exploring relevant  aspects to educational context. In this sense, evaluate the products considering variables such as usability and effectiveness should provide significant results for this study. In addition, the resulting mobile applications can be applied in real scenarios of teaching and learning, in order to evaluate the M-SPLear\allowbreak ning applied to your potential users.

%Outra perspectiva interessante está relacionada à investigação das outras estratégias de adoção propostas por. O modelo extrativo poderia ser aplicado a produtos externos a M-SPLear\allowbreak ning, com o objetivo de expandir suas features. Além disso, o modelo reativo poderia ser estudado como uma alternativa para evoluções futuras da LPS proposta.
We also intend to investigate the use of other adoption models \cite{krueger02}. For instance, the extractive model can be applied in similar products to those generated by M-SPLear\allowbreak ning aiming at increasing the validity of similarities and variabilities specified. The reactive model can be investigated as an alternative to the evolution of the proposed SPL as well.